%% ------------------------------------------------------------------
%%  Methods Section – Da‑P_Satulon with Saturon Integration
%%  This file lives in paper_G1/sections/methods.tex
%%  ------------------------------------------------------------------
\section{Methods\label{sec:methods}}

This section describes the computational framework for studying \saturon{} 
network dynamics in 3D cellular automata, including the GPU acceleration 
and statistical analysis methods that enable systematic testing of \posp{} 
predictions.

%--------------------------------------------------------------------
\subsection{3D Saturon Network Model}
%--------------------------------------------------------------------

We implement the \posp{} framework using a cubic lattice 
$\Lambda = \{(i,j,k) : 1 \leq i,j,k \leq L\}$ with $N = L^3$ discrete 
spacetime cells. Each cell's occupation state $\sigma_{i,j,k}(t) \in [0,1]$ 
represents the local \saturon{} field density, evolving according to the 
\posp{}-based update rule:

\begin{equation}
\sigma_{i,j,k}(t+1) = (1-\rho)\sigma_{i,j,k}(t) + \frac{\rho}{6}\sum_{\langle \ell,m,n \rangle} \sigma_{\ell,m,n}(t)
\label{eq:saturon_update}
\end{equation}

where the sum runs over the six nearest neighbors (representing 6-connected 
\saturon{} interactions) and $\rho \in [0,1]$ controls the \saturon{} coupling 
strength. Zero-flux boundary conditions maintain \saturon{} conservation at 
all faces of the cubic domain.

This discrete update rule captures the essential \posp{} physics: when 
$\rho \rightarrow 1$, neighboring cells approach occupation saturation, 
creating topological defects (\saturon{} particles) that mediate long-range 
information transfer.

%--------------------------------------------------------------------
\subsection{Saturon-Mediated Information Conductivity}
%--------------------------------------------------------------------

We quantify \saturon{} network dynamics using three complementary measures 
that probe different aspects of topological defect behavior:

\paragraph{Simple Conductivity (Direct Saturon Propagation):}
\begin{equation}
C_{\text{simple}}(t) = \frac{1}{N}\sum_{i,j,k} \sigma_{i,j,k}(t)
\label{eq:simple}
\end{equation}

This measures the average \saturon{} field density, capturing basic propagation 
through the network without sensitivity to spatial coherence.

\paragraph{Entropy-based Conductivity (Saturon Coherence):}
The Shannon entropy of the \saturon{} field distribution $p_s = |\{(i,j,k): \sigma_{i,j,k} \in [s, s+\Delta s)\}|/N$:
\begin{equation}
C_{\text{entropy}}(t) = -\sum_{s} p_s(t) \log p_s(t)
\label{eq:entropy}
\end{equation}

This captures \saturon{} coherence preservation across spatial domains, 
showing strong sensitivity to topological network structure.

\paragraph{Gradient-based Conductivity (Saturon Network Criticality):}
Spatial gradient magnitude of the \saturon{} field:
\begin{equation}
C_{\text{gradient}}(t) = \frac{1}{N}\sum_{i,j,k} \sqrt{(\nabla \sigma)_{i,j,k}^2}
\label{eq:gradient}
\end{equation}

This probes \saturon{} network criticality by detecting sharp spatial variations 
that signal phase transition boundaries.

%--------------------------------------------------------------------
\subsection{GPU-Accelerated Saturon Dynamics}
%--------------------------------------------------------------------

Large-scale \saturon{} network simulations ($L \geq 30$) require substantial 
computational resources. We implement GPU acceleration using CuPy, enabling 
parallel processing of \saturon{} field updates across the entire 3D grid:

\begin{algorithm}[t]
\caption{GPU-accelerated 3D Saturon network evolution}
\label{alg:gpu_saturon}
\begin{algorithmic}[1]
\STATE \textbf{Input:} \saturon{} field $\sigma^{(t)} \in \mathbb{R}^{L \times L \times L}$, coupling $\rho$
\STATE \textbf{Output:} Updated \saturon{} field $\sigma^{(t+1)}$
\STATE Copy $\sigma^{(t)}$ to GPU memory
\STATE Launch CUDA kernel with thread blocks covering $(i,j,k)$ indices
\FOR{each thread $(i,j,k)$ in parallel}
    \STATE Compute \saturon{} neighbor sum: $s = \sum_{\langle \ell,m,n \rangle} \sigma_{\ell,m,n}^{(t)}$
    \STATE Apply \posp{} update: $\sigma_{i,j,k}^{(t+1)} = (1-\rho)\sigma_{i,j,k}^{(t)} + \rho s/6$
\ENDFOR
\STATE Copy result back to CPU memory
\end{algorithmic}
\end{algorithm}

The implementation automatically falls back to CPU execution when GPU resources 
are unavailable, ensuring broad accessibility for \posp{} research.

%--------------------------------------------------------------------
\subsection{Saturon Network Optimization}
%--------------------------------------------------------------------

We employ Bayesian optimization using Optuna~\cite{optuna2019} to identify 
optimal \saturon{} coupling strengths. The objective function maximizes 
entropy-based conductivity to probe \saturon{} coherence:

\begin{equation}
\rho^* = \arg\max_{\rho \in [0,1]} \mathbb{E}[C_{\text{entropy}}(T)]
\label{eq:saturon_optimization}
\end{equation}

where $T$ is the final simulation time and the expectation averages over 
random \saturon{} field initializations. Tree-structured Parzen Estimator 
(TPE) acquisition guides the search with 50 optimization trials.

%--------------------------------------------------------------------
\subsection{Saturon Network Critical Point Detection}
%--------------------------------------------------------------------

\saturon{} network phase transitions are identified by locating maximum 
susceptibility in the conductivity response:

\begin{equation}
\rho_c = \arg\max_{\rho} \left|\frac{\partial C_{\text{saturon}}(\rho)}{\partial \rho}\right|
\label{eq:saturon_critical}
\end{equation}

We perform systematic parameter sweeps over $\rho \in [0.05, 0.20]$ with 
$N_{\text{sweep}}=10$ uniformly spaced points. The gradient is estimated 
using central differences with binomial smoothing to reduce numerical noise 
in the \saturon{} field dynamics.

%--------------------------------------------------------------------
\subsection{Statistical Analysis for POSP Verification}
%--------------------------------------------------------------------

\paragraph{Bootstrap Confidence Intervals:}
We generate $N_{\text{bootstrap}}=1000$ resampled datasets to estimate 95\% 
confidence intervals for all \saturon{} network observables, enabling robust 
statistical verification of \posp{} predictions.

\paragraph{Finite-Size Scaling for Saturon Networks:}
Critical exponents for \saturon{} network percolation are extracted using:
\begin{equation}
C_{\text{saturon}}(L, \rho) = L^{-\beta/\nu} f_{\text{POSP}}\left(L^{1/\nu}(\rho - \rho_c)\right)
\label{eq:saturon_fss}
\end{equation}
where $\beta$ and $\nu$ are critical exponents specific to \saturon{} network 
phase transitions and $f_{\text{POSP}}$ is the universal scaling function 
predicted by \posp{} theory.

\paragraph{Universality Class Analysis:}
Comparison with known 3D percolation exponents ($\beta = 0.4181$, $\nu = 0.8765$) 
and 3D Ising exponents ($\beta = 0.3265$, $\nu = 0.6301$) helps classify 
the \saturon{} network universality class and test \posp{} theoretical predictions.

%--------------------------------------------------------------------
\subsection{Computational Setup for POSP Research}
%--------------------------------------------------------------------

All \saturon{} network simulations use:
\begin{itemize}
\item Grid sizes: $L \in \{25, 30, 50\}$ (3D) for finite-size scaling studies
\item Time steps: $\Delta t = 50$ for standard runs, up to $200$ for critical analysis
\item Random seeds: Fixed at 42 for reproducible \posp{} testing
\item Hardware: NVIDIA RTX series GPUs with CUDA compute capability $\geq 7.0$
\item Software: Python 3.9+, CuPy 12.0+, NumPy 1.21+ for \saturon{} dynamics
\end{itemize}

The complete computational pipeline, from \saturon{} field initialization 
through \posp{} statistical analysis, is automated and version-controlled 
to ensure reproducible verification of discrete spacetime theories.

%% ------------------------------------------------------------------
%%  End of file
%% ------------------------------------------------------------------
