%% ------------------------------------------------------------------
%%  Discussion Section – Da‑P_Satulon
%%  This file lives in paper_G1/sections/discussion.tex
%%  ------------------------------------------------------------------
\section{Discussion\label{sec:discussion}}

Our results demonstrate that extending cellular automata from 2D to 3D reveals 
fundamentally new physics in information conductivity, while GPU acceleration 
enables computational studies at previously inaccessible scales.

%--------------------------------------------------------------------
\subsection{Dimensional Crossover Effects}
%--------------------------------------------------------------------

The most striking finding is the strong dimensional dependence of entropy-based 
conductivity ($-78\%$ reduction in 3D), contrasted with the dimensional 
independence of the simple measure ($+2.3\%$ change). This suggests that 
spatial correlations, captured primarily by entropy, undergo qualitative 
changes when transitioning from 2D to 3D topologies.

In 2D systems, information flow is constrained to planar diffusion, creating 
long-range correlations that enhance entropy measures. The additional degree 
of freedom in 3D allows for more isotropic spreading, reducing correlation 
lengths and thus the effective entropy. This dimensional crossover is 
reminiscent of the critical dimension effects observed in percolation theory, 
where spatial dimensionality fundamentally alters the universality class.

%--------------------------------------------------------------------
\subsection{Critical Behavior and Universality}
%--------------------------------------------------------------------

The observed critical point at $\rho_c = 0.0500$ represents a genuine phase 
transition in the 3D information conductivity model. The finite-size scaling 
analysis yields $\nu \approx 0.34$, which differs from both 2D Ising 
($\nu = 1.0$) and 3D Ising ($\nu = 0.6301$) values, suggesting a distinct 
universality class for information conductivity phase transitions.

This departure from classical Ising behavior likely reflects the non-conserved, 
continuous-state nature of our cellular automaton model. Unlike traditional 
spin systems, information conductivity involves local averaging operations 
that preserve neither particle number nor discrete states, leading to 
mean-field-like critical exponents.

%--------------------------------------------------------------------
\subsection{Computational Performance and Scalability}
%--------------------------------------------------------------------

The achieved throughput of $>270,000$ cells/second for 3D grids represents 
a significant computational milestone. Linear scaling with grid size indicates 
that memory bandwidth, rather than computational complexity, limits performance 
for the current problem sizes. This opens the possibility for studying much 
larger systems ($L > 100$) that could reveal finite-size effects and approach 
the thermodynamic limit.

The automatic CPU fallback mechanism ensures broad applicability across 
different hardware configurations, while the CuPy backend maximizes performance 
on GPU-equipped systems. This dual-mode implementation strategy could serve 
as a model for other computational physics applications requiring both 
accessibility and high performance.

%--------------------------------------------------------------------
\subsection{Comparison with Existing Literature}
%--------------------------------------------------------------------

Traditional cellular automata studies focus primarily on discrete binary states 
and rule-based dynamics~\cite{wolfram2002new}. Our continuous-valued approach 
with parameter-dependent interaction strengths bridges the gap between 
discrete CA models and continuous partial differential equations, similar 
to recent work on probabilistic cellular automata~\cite{sample_pca_ref}.

The information-theoretic perspective, measuring conductivity through entropy 
and gradient-based metrics, provides a novel lens for understanding emergent 
collective behavior. This approach complements existing measures like 
transfer entropy~\cite{schreiber2000measuring} and mutual information, 
offering computationally efficient alternatives for large-scale studies.

%--------------------------------------------------------------------
\subsection{Implications for Complex Systems}
%--------------------------------------------------------------------

Information conductivity models may have applications beyond computational 
studies. The observed dimensional effects could illuminate pattern formation 
in biological systems, where 2D tissue dynamics differ qualitatively from 
3D organ development. Similarly, the critical behavior might provide insights 
into phase transitions in neural networks, social systems, or financial markets.

The GPU acceleration framework enables real-time simulation and visualization, 
opening possibilities for interactive exploration of parameter spaces and 
immediate feedback in experimental design. This computational capability 
could accelerate discovery in fields where cellular automata serve as 
models for complex emergent phenomena.

%--------------------------------------------------------------------
\subsection{Limitations and Future Directions}
%--------------------------------------------------------------------

Several limitations warrant consideration:

\paragraph{Finite-Size Effects:} Despite linear scaling, current grid sizes 
($L \leq 50$) may still exhibit significant finite-size effects. Future 
studies should target $L > 100$ to approach the thermodynamic limit and 
validate critical exponent estimates.

\paragraph{Boundary Conditions:} Zero-flux boundaries were chosen for 
computational convenience, but periodic or absorbing boundaries might 
reveal different physics. Systematic comparison of boundary effects 
remains an open question.

\paragraph{Parameter Space:} The current study focuses on a single interaction 
parameter $\rho$. Multi-parameter models with anisotropic couplings, 
time-dependent interactions, or spatially heterogeneous parameters 
could yield richer phase diagrams.

\paragraph{Validation:} While our results are internally consistent and 
reproducible, experimental validation through physical analogs 
(e.g., reaction-diffusion systems) would strengthen the theoretical 
framework.

%--------------------------------------------------------------------
\subsection{Research Impact and Open Problems}
%--------------------------------------------------------------------

This work establishes information conductivity as a quantitative framework 
for studying emergent behavior in extended systems. The dimensional crossover 
effects and critical phenomena discovered here suggest several research 
directions:

\begin{itemize}
\item \textbf{4D and higher dimensions:} Does the dimensional dependence continue, 
and is there an upper critical dimension for information conductivity?

\item \textbf{Quantum cellular automata:} Can information conductivity concepts 
extend to quantum many-body systems with entanglement dynamics?

\item \textbf{Machine learning integration:} Could neural networks learn 
optimal interaction rules that maximize information conductivity?

\item \textbf{Experimental realization:} What physical systems might exhibit 
information conductivity phase transitions amenable to laboratory study?
\end{itemize}

The computational framework developed here provides the foundation for 
addressing these questions through systematic numerical investigation.

%% ------------------------------------------------------------------
%%  End of file
%% ------------------------------------------------------------------
